% !TeX program = pdfLaTeX
\documentclass[smallextended]{svjour3}       % onecolumn (second format)
%\documentclass[twocolumn]{svjour3}          % twocolumn
%
\smartqed  % flush right qed marks, e.g. at end of proof
%
\usepackage{amsmath}
\usepackage{graphicx}
\usepackage[utf8]{inputenc}

\usepackage[hyphens]{url} % not crucial - just used below for the URL
\usepackage{hyperref}
\providecommand{\tightlist}{%
  \setlength{\itemsep}{0pt}\setlength{\parskip}{0pt}}

%
% \usepackage{mathptmx}      % use Times fonts if available on your TeX system
%
% insert here the call for the packages your document requires
%\usepackage{latexsym}
% etc.
%
% please place your own definitions here and don't use \def but
% \newcommand{}{}
%
% Insert the name of "your journal" with
% \journalname{myjournal}
%

%% load any required packages here



% Pandoc citation processing
\newlength{\csllabelwidth}
\setlength{\csllabelwidth}{3em}
\newlength{\cslhangindent}
\setlength{\cslhangindent}{1.5em}
% for Pandoc 2.8 to 2.10.1
\newenvironment{cslreferences}%
  {}%
  {\par}
% For Pandoc 2.11+
\newenvironment{CSLReferences}[2] % #1 hanging-ident, #2 entry spacing
 {% don't indent paragraphs
  \setlength{\parindent}{0pt}
  % turn on hanging indent if param 1 is 1
  \ifodd #1 \everypar{\setlength{\hangindent}{\cslhangindent}}\ignorespaces\fi
  % set entry spacing
  \ifnum #2 > 0
  \setlength{\parskip}{#2\baselineskip}
  \fi
 }%
 {}
\usepackage{calc} % for calculating minipage widths
\newcommand{\CSLBlock}[1]{#1\hfill\break}
\newcommand{\CSLLeftMargin}[1]{\parbox[t]{\csllabelwidth}{#1}}
\newcommand{\CSLRightInline}[1]{\parbox[t]{\linewidth - \csllabelwidth}{#1}\break}
\newcommand{\CSLIndent}[1]{\hspace{\cslhangindent}#1}

\usepackage{lineno}
\linenumbers
\usepackage{setspace}

\begin{document}

\title{Thriving in a changing world: A 130 years shell record of an
intertidal predator from the Southern North Sea \thanks{This study was
supported by a NERC studentship awarded to DM (NE/L002507/1).} }


    \titlerunning{Thriving in a changing world}

\author{  Dennis Mayk \and  Lloyd S. Peck \and  Thierry
Backeljau \and  Elizabeth M. Harper \and  }


\institute{
        Dennis Mayk \at
     Department of Earth Sciences, University of Cambridge, Cambridge,
United Kingdom \& British Antarctic Survey, Cambridge, United Kingdom \\
     \email{\href{mailto:dm807@cam.ac.uk}{\nolinkurl{dm807@cam.ac.uk}}}  %  \\
%             \emph{Present address:} of F. Author  %  if needed
    \and
        Lloyd S. Peck \at
     British Antarctic Survey, Cambridge, United Kingdom \\
     %  \\
%             \emph{Present address:} of F. Author  %  if needed
    \and
        Thierry Backeljau \at
     Royal Belgian Institute of Natural Sciences, Brussels, Belgium \&
Evolutionary Ecology Group, University of Antwerp, Antwerp, Belgium \\
     %  \\
%             \emph{Present address:} of F. Author  %  if needed
    \and
        Elizabeth M. Harper \at
     Department of Earth Sciences, University of Cambridge, Cambridge,
United Kingdom \& British Antarctic Survey, Cambridge, United Kingdom \\
     %  \\
%             \emph{Present address:} of F. Author  %  if needed
    \and
    }

\date{Received: date / Accepted: date}
% The correct dates will be entered by the editor


\maketitle

\begin{abstract}
Ocean acidification and global climate change are predicted to
negatively impact marine calcifiers, with species inhabiting the
intertidal zone being especially vulnerable. Current predictions of
organism responses to projected changes are largely based on relatively
short to medium term experiments over periods of a few days to a few
years. Here we look at responses over a longer time span and present a
130 years shell shape and shell thickness record of the marine
intertidal predator gastropod \emph{Nucella lapillus} (dog whelk). We
show that contrary to global predictions, \emph{N. lapillus} built
continuously thicker shells and exhibits significant changes in shell
shape on the Southern North Sea coast throughout the last century. We
argue that the observed shell thickening of \emph{N. lapillus} is the
result of higher annual temperatures, longer yearly calcification
windows, near shore eutrophication and enhanced prey abundance,
mitigating the impact of global climate change.
\\
\keywords{
        shell thickness \and
        mineralogy \and
        shell shape \and
        ocean acidification \and
        invasive species \and
        eutrophication \and
        ocean warming \and
    }


\end{abstract}


\def\spacingset#1{\renewcommand{\baselinestretch}%
{#1}\small\normalsize} \spacingset{1}


\doublespacing

\hypertarget{intro}{%
\section{Introduction}\label{intro}}

With the increase of carbon dioxide in the atmosphere accompanied by a
rise in (seawater) temperature, ocean acidification (OA; decreasing
seawater pH) and more frequent extreme weather events, many marine
species and ecosystems are at risk. In recent years, much effort has
been made to study the impact of rising seawater temperatures, OA and
the synergistic mixed effects of both on the development and survival of
species at risk (Byrne 2011; Byrne and Przeslawski 2013; Fitzer et al.
2015; Hofmann et al. 2010; Kroeker et al. 2013, 2016; Parker et al.
2013). Special attention was thereby given to study the impacts on
larval development (Przeslawski, Byrne, and Mellin 2015; Suckling et al.
2015; Thomsen et al. 2015; Waldbusser et al. 2014, 2015) and
biomineralization in both juveniles and adults, especially in calcium
carbonate shell-bearing species(Barclay et al. 2019; Cross, Peck, and
Harper 2015; Cross et al. 2015) as those are believed to be most
impacted by global climate change (Byrne and Przeslawski 2013; Doney et
al. 2009; Watson et al. 2012). Despite these efforts, our understanding
of, and ability to predict future changes in global or regional
ecosystems, assemblages, populations, or even single species is still
limited.

Many relatively short-term experiments ranging from days to a few years
report predominantly negative impacts on shell strength, size and
thickness of shell forming marine species (Barclay et al. 2019; Fitzer
et al. 2014; Gaylord et al. 2011; Gazeau et al. 2013). Yet, sometimes no
such effects are observed or even reversed effects (e.g., shell
thickening) are reported (Bullard et al. 2021; Cross, Peck, and Harper
2015; Cross, Harper, and Peck 2019). Long-term experiments accounting
for phenotypic plasticity (within and across generations) and genetic
adaptations are receiving increasingly more attention (Ashton et al.
2017; Cornwall et al. 2020; Donelson et al. 2018; Gibbin et al. 2017) as
these processes are critical to allow for the full phenotypic responses
to the altered environment of the future ocean (Peck 2011; Somero 2012;
Telesca et al. 2021). Conventional laboratory experiments often lack the
variability and complexity of natural ecosystems by keeping relevant
parameters stable to reduce the experimental ``noise'' (Kroeker et al.
2020). This has the potential benefit of unmasking miniscule organism
responses that would otherwise remain unnoticed due to fluctuating
environmental factors. However, these responses may not be
representative of organisms inhabiting the natural environment, as
spatial and temporal variability of temperature and of the carbonate
system often already exceed the mean of anticipated future scenario
changes. In addition, mixing of these factors and their variation could
lead to unexpected organism responses (Kroeker et al. 2016). This raises
the important question of what influence environmental variability
through time and space has at the species, community or ecosystem level
(Kroeker et al. 2020). Early efforts to answer this question suggested a
significant effect of natural variability on individual species and the
ecosystem, emphasizing the non-linear relationship between environmental
stressors and biological response (Bernhardt et al. 2018; Harley et al.
2017; Kroeker et al. 2016). Accordingly, more complex \emph{in situ}
studies have begun to highlight the complexities of the multitude of
possible interactions between biological and environmental factors in
the mean response and variability to a changing climate system (Ashton
et al. 2017; Barnes et al. 2021). A novel approach to account for the
inherent complexity of marine ecosystems is the use of archival museum
collections to study phenotypic responses to historical changes in a
particular species or populations (Bullard et al. 2021; Cross, Harper,
and Peck 2018; Telesca et al. 2021). This approach offers a unique way
to study organism responses, as it observes outcomes that by their
nature include all natural organism-organism and organism-environment
interactions at every given time interval. This allows the
identification of organism or population responses in complex natural
systems over significantly longer time frames, allowing us to extend the
understanding gained from laboratory or in situ experiments.

The intertidal zone is a harsh and demanding environment in which
sessile or slow-moving inhabitants such as the gastropod \emph{Nucella
lapillus} (Linnaeus, 1758) (dog whelk) are regularly exposed to a range
of biotic and abiotic stressors including varying water levels, wave
action, air exposure, steep salinity gradients, large temperature
variation and changing predator regimes. Salinity levels in tidal pools
may change rapidly depending on external factors such as evaporation,
precipitation or ice encasement (Clarke and Beaumont 2020). Lacking the
ocean's heat capacity which buffers short-term temperature fluctuation,
intertidal organisms exposed to air and direct solar insolation have to
endure excessive heat which may be intensified in shelled organisms
(Clark, Peck, and Thyrring 2021; Thyrring et al. 2017). In terms of
global climate change, the intertidal zone is considered one of the most
vulnerable areas as the interaction between global temperature rise and
tidal movement may create a mosaic-like thermal environment that can be
detrimental to intertidal organisms through e.g., cardiac arrest or
protein damage (Angilletta et al. 2013; Hofmann and Somero 1995;
Roberts, Hofmann, and Somero 1997). In addition, global climate change
is expected to cause more extreme weather conditions (Easterling et al.
2000; Meehl and Tebaldi 2004; Rahmstorf and Coumou 2011; Rummukainen
2012) which will be accompanied by more regular and more intense storm
events, putting intertidal organisms at additional risk (Nehls and Thiel
1993). Intertidal organisms are adapted to hostile conditions but an
increase in the occurrence or duration of spatial and temporal stressor
extremes provoked by global climate change may exceed their tolerance
limits. In addition to the natural stressors, anthropogenic interactions
with the environment through e.g., ship traffic, communal waste
discharge, mining, tourism and agricultural land use further exacerbate
the hostility of the intertidal zone. For example, increased land use
for agricultural purposes and associated use of fertilisers resulted in
a significant nutrients burden of riverine input into the North Sea
(Brion et al. 2008). This resulted in eutrophication of the near shore
accompanied by an increase in primary production (Brion et al. 2008;
Gypens, Borges, and Lancelot 2009; Mackenzie, Ver, and Lerman 2002).

The carnivorous gastropod \emph{Nucella lapillus} is an important
predator in the intertidal zone (Burrows and Hughes 1989; Hughes and
Burrows 1994). It is highly abundant on rocky shores in the Northern
Atlantic and is, as an intertidal dweller, regularly exposed to air and
extreme wave regimes. Due to its pronounced shell plasticity along
environmental gradients such as wave energy and predator abundance
(Palmer 1990; Pascoal et al. 2012) it has been the focus of numerous
studies. Its calcareous shell consists of two distinct layers; irregular
calcite externally and crossed lamellar aragonite internally which may
be separated by a transitional spherulitic layer (Mayk 2020). The shell
exhibits a high degree of variability in morphology (Berry and Crothers
2009; Crothers 1985; Galante-Oliveira et al. 2011), apertural teeth
expression (Appleton and Palmer 1988; Cowell and Crothers 1970)(Appleton
\& Palmer, 1988; Cowell \& Crothers, 1970; Palmer, 1990) and most
notably shell thickness Pascoal et al. (2012) in accordance with
environmental stressors. These characteristics make \emph{N. lapillus} a
valuable candidate to study the impact of long-term environmental change
on intertidal calcifiers.

\hypertarget{materials-and-methods}{%
\section{Materials and Methods}\label{materials-and-methods}}

\hypertarget{sample-area-and-collection}{%
\subsection{Sample area and
collection}\label{sample-area-and-collection}}

Specimens were collected from seven sites along the Belgian and Dutch
coast (Supplementary Table 1) between 1888 and 2019 and archived at the
Royal Belgian Institute of Natural Sciences, Brussels, Belgium (RBINS).
Archival specimens were either stored dry or wet in ethanol (70\%)
filled preservation jars. The total sampling area extends over
\(\sim49\) km from the port of Oostende (51°14'N 2°55'E) across the
Westerschelde estuary to the beach of Zoutelande (51°29'N 3°28'E)
(Supplementary Figure 2a). The Belgian coast is dominated by wide sandy
beaches with little hard substratum, rendering it generally unsuitable
habitat for barnacles, mussels, or the predatory \emph{N. lapillus}.
However, the extensive construction of man-made structures against
coastal erosion, wave exposure and flooding such as groynes, port
structures, and dikes provide excellent substrata for rocky shore
organisms and allowed the development of successful populations of these
organisms on the Belgian and Dutch coast. In this study a total of 150
\emph{N. lapillus} specimens were analysed. Only archival specimens with
detailed collection date and location descriptions were included. No
samples were available between 1978 and 2019 due to the extensive use of
the organotin-compound tributyltin (TBT) as an antifouling agent on
ships that caused imposex in female \emph{N. lapillus} (Gibbs, Bryan,
and Pascoe 1991; Gibbs 1999; Oehlmann et al. 1998). The use of TBT led
to a local extinction of \emph{N. lapillus} on the Belgian coast in the
late 1970s (Kerckhof 1988), and they only reappeared in 2012 (De Blauwe
and D'acoz 2012). All specimens collected were adults (likely at sexual
maturity) with shell heights \textgreater{} 17 mm (Galante-Oliveira et
al. 2010). The average shell height of selected specimens was 28.2 ±
2.73 mm (1\(\sigma\)). Shells selected for this study had intact apices
and showed no exterior and interior signs of dissolution nor other
apparent shell damage.

\hypertarget{environmental-dataset}{%
\subsection{Environmental dataset}\label{environmental-dataset}}

To investigate changes in environmental stressors along the Belgian
coast over the last 130 years, four key environmental parameters (wind
speed, air temperature, sea surface temperature, sea surface salinity)
were investigated. Detailed descriptions of the datasets are provided in
the supplementary material. Long-term time series of monthly
measurements for wind speed, sea surface temperature and air temperature
from 1880 -- 2020 in the area of 51º -- 52º N and 2º -- 4º E were
obtained from the International Comprehensive Ocean-Atmosphere Data Set
3.0 (National Centers for Environmental Information/NESDIS/NOAA/U.S.
Department of Commerce et al. 2016). This dataset provides assimilated
original marine surface data with traceable data sources and thus
provides a reliable source with exceptionally high spatial and temporal
resolution. Data for salinity from 1961 -- 2018 were obtained from the
ICES Dataset on Ocean Hydrography, 2020 (ICES 2020). Since near-shore
salinity regimes generally exhibit a mosaic pattern due to differences
in surface and underground freshwater run-offs, salinity differences
between sampling sites are much higher than expected decadal changes. In
addition, the ICES data did not allow for the construction of reliable
time series at every sampling site, hence median salinities of all
available data points within a 10 km radius around either sampling site
are reported and compared. Inspection of the salinity data showed that
these were neither normally distributed (Shapiro Wilk test and visual
inspection with QQ plots) nor exhibited an equal variance (Levene's test
and visual inspections of the residuals). Hence, differences in salinity
between sampling sites were evaluated by a non-parametric Kruskal-Wallis
H test and a pairwise Wilcoxon Rank Sum post-hoc test with a standard
Bonferroni correction. Changes in seawater pH could not be directly
assessed as there is a lack of reliable long-term pH trends for the
(Southern) North Sea (Ostle et al. 2016). It is suggested that tracing
long term trends in the seawater carbonate system requires data sets of
at least 25 years (Fay and McKinley 2013) which are not available for
the sampling sites.

\hypertarget{shell-characteristics}{%
\subsection{Shell characteristics}\label{shell-characteristics}}

For all specimens, measurements of shell height, aperture height and
aperture width were made using digital Vernier callipers (±0.1 mm).
Shell shape was evaluated using elliptic Fourier analysis (EFA) of shell
outlines (Giardina and Kuhl 1977; Kuhl and Giardina 1982) following the
approach in Telesca et al. (2018). For this, individual shells were
placed on an illuminated platform with the aperture facing downward in a
natural position and images were taken from a fixed distance with a DSLR
camera (Nikon 3000 with a Sigma 105 mm macro lens) mounted on a
photostand. This produced images of the shells as black silhouettes with
sharp edges on a bright background making them suitable for outline
tracing. We constructed the outlines in a natural position (i.e.,
aperture facing downward) rather than in the more commonly used aperture
upwards position to study shape changes that are meaningful to \emph{N.
lapillus} in a natural setting. Outlines of individual shells were
digitised and turned into x-y coordinates using the R package Momocs
(Bonhomme et al. 2014; R Core Team 2020). Outlines were subsequently
smoothed to remove digitisation noise and harmonised regarding
orientation and size by Procrustes analysis. An EFA was then carried out
on the aligned outlines with nine harmonics encompassing 99\% of the
total harmonic power and a principal component analysis (PCA) was used
to extract axes that reflect most of the shell shape variability among
individuals. Shell layer thickness was analysed by sectioning the shells
perpendicular to the aperture at the mid-section of the last whorl using
a diamond saw (Figure 1a). The anterior side was embedded in polyester
resin (Kleer-Set FF, MetPrep, Coventry, UK). The intersection plane of
embedded specimens was smoothed on silicon carbide paper and
subsequently polished down to one micron with diamond paste using an
automated polishing rig. Micrographs of the shell sections were taken
from the dorsal region (Figure 1a) of the shell halves using a digital
stereo microscope (ZEISS Primo Star, Zeiss, Oberkochen, Germany). Since
aperture thickness is influenced by both predation pressure and wave
exposure in \emph{N. lapillus} through the formation of thicker
apertural lips and/or teeth (Appleton and Palmer 1988; Avery and Etter
2006; Palmer 1990; Pascoal et al. 2012), measurements of shell layer
thickness further inside the shell provides a better proxy for
environmentally induced shell layer thickness variability. Aragonite and
calcite layer thicknesses were individually measured from 10 randomly
chosen regions within the micrographs using the software ZEISS Labscope
(v.3.1). For this purpose, ten approximately equally distanced sections
were selected on either micrograph and total shell thickness and calcite
layer thickness were measured perpendicular to the innermost aragonite
layer of the shells. Aragonite layer thickness was inferred by
subtracting calcite layer thickness from total shell thickness. To
validate the reproducibility of this method the same images were
evaluated three times in different orders and on different days showing
no statistically significant difference between measurement sessions
(Kruskal-Wallis H test, H(2) = 1.029 \emph{p} = 0.598).

\begin{figure}
\centering
\includegraphics{../plots/figure_1.pdf}
\caption{Shell layer thickness trends in archived \emph{Nucella
lapillus} from the Belgian and Dutch coasts over the last 130 years. (a)
Indication of a shell section used for the thickness analysis. The
dorsal region where thickness measurements were carried out is
highlighted in red. (b) GAM thickness trend of the crossed lamellar
aragonite layer (red) and thickness trend of the irregular prismatic
calcite layer (blue). (c) GAM shell height (green) and aperture size
(brown) trends. 95\% confidence intervals are represented by shaded
areas.}
\end{figure}

\hypertarget{environmental-timeseries-analysis}{%
\subsection{Environmental timeseries
analysis}\label{environmental-timeseries-analysis}}

Long term time series (1880 - 2020) of sea surface temperature, air
temperature and wind speed were modeled with generalised additive mixed
models (GAMM) using the R package mgcv (Wood 2011) and following
protocols for data exploration and regression-type analyses (Zuur, Ieno,
and Elphick 2010; Zuur and Ieno 2016). Each dependent variable was
modeled using month and year as fixed covariates and their tensor
product interaction term. The final model was of the form:

\[
ED_i = \beta_0 + f(month_i) + f(year_i) + f(month_i, year_i) + \epsilon_i
\] \[
\epsilon_i \sim N(0; \sigma^2)
\]

Where \(ED_i\) is the \(i^{th}\) observation of the time series,
\(\beta_0\) is the intercept, \(f(month_i)\) and \(f(year_i)\) are
smooth functions of months and years, \(f(month_i, year_i)\) is the
interaction term of the two variables and \(\epsilon_i\) is the normally
distributed and correlated error term. For both smooth terms thin plate
regression splines were used, and the number of knots for months was
limited to 12. Plots of the partial autocorrelation function showed
significant autocorrelation of the residuals which required the use of
autoregressive models to account for the temporal autocorrelation. To
identify significant trend changes in the modeled time series the first
derivative of the fitted model was calculated using finite differences.
Since the equation of the regression splines is unknown, finite
differences provide a method to resample a fitted spline with a chosen
number of points. This enabled us to calculate the first derivative of a
fitted regression line through these resampled points which resembles a
close approximation of the fitted spline.

We also estimated yearly calcification windows (i.e., days per year when
mean temperatures reach a certain threshold) from monthly mean and
standard deviation temperature. It is well known that biomineralization
rates are halted or greatly reduced when the environmental temperature
falls below an organism or population specific temperature threshold
(Doroudi, Southgate, and Mayer 1999; Joubert et al. 2014; Laing 2000).
In laboratory experiments \emph{N. lapillus} from the coast of Whitby,
UK showed a calcification threshold temperature of \(\sim13\) °C (below
13 °C no noticeable calcification occurred; D.M. personal observation).
To account for possible differences in temperature tolerances between
\emph{N. lapillus} populations in the UK and the sampling site on the
Southern North Sea coast, yearly calcification windows were estimated
for temperature thresholds of 10 °C to 15 °C in 1 °C increments (Figure
2). Number of days per year exceeding the threshold temperature were
then modeled using generalized additive models (GAM) of the form:

\[
CTD_i = \beta_0 + f(year_i) * ct + ct + no + \epsilon_i
\] \[
\epsilon_i \sim N(0; \sigma^2)
\]

Where \(CTD_i\) is the \(i^{th}\) observation of the time series,
\(\beta_0\) is the intercept, \(f(year_i)\) is the smooth function of
years, \(ct\) represents the selected calcification thresholds and
\(no\) refers to the number of observations per year.

\hypertarget{shell-characteristics-timeseries-analysis}{%
\subsection{Shell characteristics timeseries
analysis}\label{shell-characteristics-timeseries-analysis}}

Selected shell shape and thickness descriptors, namely shell height,
aperture size, shape-PC1 - shape-PC4, calcite layer thickness and
aragonite layer thickness were modeled by GAMs to identify significant
changes among sampling locations, shell size and through time. GAMs
included years to model long-term trends among shell descriptors, shell
height to account for morphological and compositional variability caused
by a difference in size among individuals which was important to avoid
biases due to allometric scaling, and sampling site as random-effect to
account for sampling sites specific effects such as wave, salinity or
predation exposure on the dependent variable. The final model was of the
form:

\[
SD_{ij} = \beta_0 + f(year_i) + f(shell\:height_{ij}^*) + site_i + \epsilon_{ij}
\] \[
site_i \sim N(0; \sigma^2)
\] \[
\epsilon_{ij} \sim N(0; \sigma^2)
\]

Where \(SD_j\) is the \(j^{th}\) observation of the shell descriptor in
site i (i = 1-7), \(\beta_0\) is the intercept, \(f\) are the smoothing
functions, \(site_i\) is the random-effect and \(\epsilon_i\) is the
normally distributed error term. The asterisk denotes that this term was
not included in the GAM modeling shell height trends.

\hypertarget{global-timeseries-analysis}{%
\subsection{Global timeseries
analysis}\label{global-timeseries-analysis}}

The effect of long-term environmental change in wind speed, sea surface
temperature, and air temperature on the selected shell characteristics
were analysed by a sequence of global models. Therefore, descriptive
statistics for each environmental variable were calculated (i.e., mean,
median, min, max, \(10^{th}\) percentile, \(25^{th}\) percentile,
\(75^{th}\) percentile, \(90^{th}\) percentile and standard deviation).
To account for the natural variability spectrum that individual
specimens experienced throughout their lifetime the descriptive
statistics were calculated for a time window encompassing three years up
to the sampling year of the respective samples. A PCA was then performed
on the normalized descriptive statistics for each environmental
parameter to describe temperature and wind speed ``regimes'' (Figure 2)
(Kroeker et al. 2016; Telesca et al. 2021). This provides the benefit of
reducing dimensionality in the data and allows to model complex
relationships while avoiding collinearity among explanatory variables.
GAMs were then constructed for each selected shell characteristic
including the first two principal components of each environmental
parameter as independent variables. An automated variable selection
process using the ``preferred'' double penalty approach (Marra and Wood
2011) was used to select covariables with the most influence on the
response variable. Sampling site was included in all models as random
effect.

\begin{figure}
\centering
\includegraphics{../plots/figure_2.pdf}
\caption{Biplots of sea surface temperature, air temperature and wind
speed regimes at the Belgian coast from 1890 until 2019. (a) Biplot of a
principal component analysis of descriptive statistics for sea surface
temperature covering 66.2 \% of the total variability. (b) PCA biplot of
air temperature covering 74.7 \% of the total variability and (c) PCA
biplot of wind speed covering 93.2 \% of the total variability.}
\end{figure}

\hypertarget{results}{%
\section{Results}\label{results}}

\hypertarget{environmental-change}{%
\subsection{Environmental change}\label{environmental-change}}

Long-term models of selected environmental parameters reveal an increase
in air temperature, sea surface temperature and wind speed from 1880 to
2020 (Figure 3, Supplementary Table 2). Mean air temperature increased
significantly (\(F_{1}\) = 17.41, p \textless{} 0.001) by 1 ºC. Mean sea
surface temperature increased significantly (\(F_{4.59}\) = 25.78, p
\textless{} 0.001) by 2 ºC with a rapid acceleration of warming since
1985 to the present. The winter months of December to February exhibit
the most pronounced increase in temperature throughout the time-series
(Supplementary Figure 4). Wind speed exhibited an initial drop from 1880
to 1930 by about 2 \(ms^{-1}\) after which it rose significantly
(\(F_{8.27}\) = 40.9, p \textless{} 0.001) by 3 \(ms^{-1}\).

\begin{figure}
\centering
\includegraphics{../plots/figure_3.pdf}
\caption{Historical trends of (a) air temperature, (b) sea surface
temperature and (c) wind speed within the sampling region from 1880 to
2020. For each environmental parameter both the overall change including
all individual data points (bottom panels) and the temporal change
centred around the maximum variability (top panel) are presented.
Coloured lines mark periods of significant increases (red) or decrease
(blue). 95\% confidence intervals (shaded area) are reported for each
predictor (dashed lines).}
\end{figure}

Salinity data comparison by Kruskal-Wallis rank sum test confirmed the
anticipated significant inhomogeneity between sampling sites (H(6) =
832.414 \emph{p} = \textless{} 0.001). A pairwise Wilcoxon rank sum test
with Bonferroni adjusted p-values for a pairwise comparison of median
salinities between each sampling site (Supplementary Table 3) reported
that the median salinity of 30.1 psu at Zwarte Polder was significantly
lower than at the other sites. Median salinities at Blankenberge (30.9
psu), Zeebrugge (30.9 psu), Duinbergen (30.8 psu) and Knokke (30.9 psu)
were not significantly different due to the proximity of these sites.
Oostende exhibited a median salinity of 32.7 psu which is 2.6 psu higher
than at Zwarte Polder and significantly higher than at all other sites.

Calcification windows significantly increased non-linear from 1880 until
2019 irrespective of selected temperature threshold (Figure 4, Table 1).
The percentage increases of days per year above temperature thresholds
(10 °C -- 15 °C) range between 27 \% to 123 \%. The biggest percentage
increase was found in the 15 °C temperature threshold group which means
that days reaching sea surface temperatures of 15 °C occurred more than
double as often in 2019 than in 1880.

\begin{figure}
\centering
\includegraphics{../plots/figure_4.pdf}
\caption{Calcification window change estimates. (a) Historical trends of
estimated calcification window length per year expressed as the number
of days in which the sea surface temperature surpassed a selected
threshold. 95\% confidence intervals (shaded area) are reported for each
predictor (dashed lines). (b) Absolute and relative change in days
surpassing temperature threshold between 1890 and 2019.}
\end{figure}

\hypertarget{long-term-trends-of-shell-formation}{%
\subsection{Long-term trends of shell
formation}\label{long-term-trends-of-shell-formation}}

Shell shape PCs of \emph{N. lapillus} significantly changed with
sampling year (shape-PC1: \(F_{1}\) = 13.25, p \textless{} 0.001;
shape-PC2: \(F_{1.98}\) = 4.1, p = 0.021; shape-PC4: \(F_{1}\) = 5.26, p
= 0.023) (Figure 5, Supplementary Figure 1). The first four principal
components (shape-PC1 - shape-PC4) capture 94 \% of the total shape
variance among specimens. The most pronounced shell shape variability
between individuals captured by shape-PC1 is an inclination of the
longitudinal axis whereby variability captured by shape-PC2 describes a
rounding of the shell. Shape-PC3 describes the shape of the shells'
apices and shape-PC4 describes the angularity of the shells (Figure 5).
Unlike the other shape descriptor, shape-PC3 did not significantly
change over the past 130 years and was therefore omitted from further
analysis. Despite their significance, the deviance explained by the
shape GAMs is low in all models (\(R^2\): 12.6 - 15.9), suggesting that
other unaccounted factors had a strong influence on the shell shape of
the herein studied \emph{N. lapillus} specimens. All models performed
significantly better than the respective intercept-only model.

\begin{figure}
\centering
\includegraphics{../plots/figure_5.pdf}
\caption{Shell shape trends of significant shape descriptors (a)
shape-PC1, (b) shape-PC2 and (c) shape-PC4 in the intertidal gastropod
\emph{Nucella lapillus} from the Belgian and Dutch coast collected
between 1888 and 2019. Shape variance (\(\overline{x} \pm 3\sigma\))
represented by either shape-PC is presented in (d).}
\end{figure}

Total shell layer thickness significantly increased from 1888 to 2019.
Aragonite layer thickness significantly increased by 0.1 mm (\(F_{1}\) =
6.08, p = 0.015), i.e., by 79 \%. Calcite layer thickness also increased
significantly by 0.22 mm (\(F_{1}\) = 3.91, p = 0.05), i.e., by 38\%.
Aperture size increased significantly between 1888 and 2019
(\(F_{1.75}\) = 5.24, p = 0.016) by 14.2 \(mm^2\), i.e., by 10 \%. Mean
shell height significantly increased by up to 19 \% (\(F_{1.85}\) =
24.13, p \textless{} 0.001) (Figure 1). Shell layer thickness, shell
height and aperture size models performed significantly better than the
intercept-only models. GAM summary statistics of shell shape and
thickness models are provided in Supplementary Table 4.

\hypertarget{environmental-impact-on-shell-calcification}{%
\subsection{Environmental impact on shell
calcification}\label{environmental-impact-on-shell-calcification}}

Global GAMs with automated model term selection suggest varying
influences of environmental regimes on selected shell characteristics
(Figure 6). Shell height and aperture size are significantly influenced
by air temperature, sea surface temperature and wind speed. Aragonite
layer thickness in the studied specimens significantly correlated with
sea surface temperature and wind speed. Calcite layer thickness is
significantly influenced only by air temperature. All shape-PCs
correlated significantly with sea surface temperature. Furthermore,
shape-PC2 correlated significantly with wind speed and shape-PC4
correlated significantly with air temperature. Among selected shell
descriptors only aragonite layer thickness and shell height varied
significantly with sampling sites. GAM summary statistics of global
models are provided in Supplementary Table 5.

\begin{figure}
\centering
\includegraphics{../plots/figure_6.pdf}
\caption{Correlation matrix of global model outputs for selected shell
characteristics. Y-axis shows model terms and x-axis represents the
dependent variables (shell shape and thickness descriptors). Color
intensity reflects significance of correlation given by the models'
\emph{p}-values; with darker colours representing lower \emph{p}-values.
Non-significant model terms (\emph{p} \textgreater{} 0.05) are
represented by blanks. Floating bar shows the deviance explained per
model in percent. Color intensity reflects the amount of deviance
explained, with darker colours referring to higher percentage of
deviance explained.}
\end{figure}

\hypertarget{discussion}{%
\section{Discussion}\label{discussion}}

Contrary to expected species responses to global climate change (Gazeau
et al. 2013), we found significant increases in shell layer deposition
in archival \emph{N. lapillus} from the Belgian and Dutch North Sea
coast from 1888 until 2019. In relative terms, aragonite layer thickness
increased twice as much as calcite layer thickness although in absolute
terms calcite thickness increased more (Figure 1). Similar shell layer
thickening was also reported in the blue mussel \emph{Mytilus edulis}
over the last century that had been sampled from adjacent regions
(Telesca et al. 2021). Unlike reported apertural lip thickening observed
in archival \emph{N. lapillus} from the US east coast that scaled solely
with shell height (Fisher et al. 2009), our shell thickness descriptors
showed only weak allometric relationships and correlated significantly
with sampling years (Figure 1, Supplementary Table 4). We also found
significant shell shape changes as well as aperture size and shell
height increases over the last 130 years. Most noticeable, shape changes
included a shift in the shells' longitudinal axis that correlated
significantly with sea surface temperature and explained almost 80\% of
the total shape variance (Figure 5 and 6). However, a similar
association between axis orientation and temperature has not been
presented in the literature so far and it is at this point unclear if
the link between the two is causal. Few studies have so far investigated
shell size changes in \emph{N. lapillus} of decade or century long time
frames and available studies present conflicting results. For example, a
study of archival \emph{N. lapillus} from the US east coast reports a
century long shell height increase (Fisher et al. 2009) but another
study of archival \emph{N. lapillus} from the Southern UK coast reports
significant shell height decreases. Generally, museum collections
complicate the interpretation of \emph{absolute} changes due to possible
sample collection biases. However, both conflicting studies had access
to exceptionally large sample sizes mitigating the risk for collection
biases which suggests a possible locality effect. In fact, GAMs (Figure
6) suggest a strong relationship between shell height and sampling
location, possibly driven by differences in wave exposure or predation
regimes (Crothers 1983, Colton 1922, Colton 1916, Berry 2009.

\hypertarget{constraints-on-shell-biomineralization}{%
\subsection{Constraints on shell
biomineralization}\label{constraints-on-shell-biomineralization}}

Shell formation through calcium carbonate precipitation is most
influenced by three factors: temperature (Clarke 1993; Watson et al.
2012), ion availability (\(Ca^{2+}\) and \(HCO_3^-\)) (Sanders et al.
2021) and the nutritional state of the organism, as shell formation
comes with metabolic energy cost (Palmer 1983, 1992; Watson, Morley, and
Peck 2017). Estimates for the amount of energy required to produce a
unit of shell varies largely between studies but may be as large as
\(\sim60\) \% of the available assimilated energy in e.g., blue mussels
from the Baltic Sea and thus can contribute significantly to an
organism's energy balance (Sanders et al. 2018). Low temperatures
decrease precipitation rates of calcium carbonates (Burton and Walter
1987) and may impair important cellular processes rendering warmer
temperatures more preferential for shell formation. Low calcium and/or
(hydrogen-) bicarbonate ion availability, which are the building blocks
of calcium carbonate may also impair shell production rates. For
example, low calcium ion concentrations in the Baltic Sea impair shell
formation in mytilid mussels (Sanders et al. 2021) and OA causes a
change in the seawater carbonate system shifting the equilibrium towards
carbonic acid, also impeding the formation of calcium carbonates. High
latitude calcifiers mitigate these limitations by incorporating a larger
proportion of organic material into their shell which in turn comes with
an increase in metabolic cost per unit shell (Palmer 1992; Sanders et
al. 2018; Telesca et al. 2019). Stress due to increasing wave action or
increasing predator abundance may also affect shell formation rates.
\emph{Nucella lapillus} responds to enhanced predator abundance by
making thicker apertural lips and increasing apertural teeth abundance
(Appleton and Palmer 1988). A study of shell layer thickening in
archival blue mussels from adjacent sites on the Belgian coast
identified significant increases in shell thickness with enhanced
predation pressure starting in the 1990s (Telesca et al. 2021). However,
since \emph{N. lapillus} was absent during this time and we specifically
sampled shell layer thickness from the dorsal region of the shell and
not from the apertural lip to control for predator stress induced
thickening of the aperture lip, it is unlikely that an increase in
predation pressure caused the observed increase in shell layer thickness
in this study. Instead, the historical thickening of \emph{N. lapillus}
shell reported here is likely coupled to a general improvement of
abiotic and biotic conditions on the Southern North Sea coast which
mitigated ocean acidification effects and thereby promoted
biomineralization.

\hypertarget{warmer-annual-temperatures-improved-calcification-conditions}{%
\subsubsection{Warmer annual temperatures improved calcification
conditions}\label{warmer-annual-temperatures-improved-calcification-conditions}}

GAM shell thickness models report a significant correlation between
temperature and shell layer thickness. Calcite layer thickness is
significantly correlated with air temperature, whereas aragonite layer
thickness significantly changes with sea surface temperature
(Supplementary Table 5). Fittingly, both sea surface temperature and air
temperature significantly increased by 1 °C and 2 °C, respectively
throughout the last 130 years along the Belgian coast (Figure 3), in
line with global climate change predictions. An increase in mean
temperature therefore likely improved calcification conditions for
\emph{N. lapillus}, enhancing precipitation rates (Burton and Walter
1987; Joubert et al. 2014; Laing 2000), and simultaneously reducing the
cost in metabolic energy per unit shell due a proportional reduction of
shell organic matter (Palmer 1983, 1992; Sanders et al. 2018). In
addition, within-year analyses of temperature time series show that
winter temperatures (both air and sea surface) increased more rapidly
than in the other seasons indicating increasingly milder winters
(Supplementary Figure 4). In the laboratory, \emph{N. lapillus} ceases
to repair shell damage or to build noticeably new shell below 13 °C
(D.M. personal observations). Milder winters could thus allow for a
longer period of biomineralization during the year, resulting in
increased shell layer thickness due to prolonged ``calcification
windows.'' In fact, calcification window estimates for temperatures from
10 -- 15 °C showed a significant increase in days surpassing the
temperature thresholds. For a temperature threshold of e.g., 13 °C a
relative increase of 58 \% was observed which in absolute terms equals
63 days. Thus, in 2019, \emph{N. lapillus} had 63 more days to calcify
than in 1880 which likely resulted in thicker shell layers through
secondary shell layer thickening. Secondary shell layer thickening
(i.e., thickening of the existing shell) only occurs in the crossed
lamellar layer in \emph{N. lapillus} which means that prolonged
calcification windows (i.e., time as a factor for enhanced
calcification) would solely influence aragonite layer thickness. We thus
hypothesise that the increase in annual sea surface and air temperature
on the Southern North Sea coast has impacted shell calcification of
\emph{N. lapillus} in two ways: (1) Acceleration of shell precipitation
rate by reducing the energetic cost per unit shell. (2) Extension of
annual calcification windows allow for enhanced secondary thickening of
the shell aragonite layer.

\hypertarget{near-shore-eutrophication-mitigated-oa-effects}{%
\subsubsection{Near shore eutrophication mitigated OA
effects}\label{near-shore-eutrophication-mitigated-oa-effects}}

Like ambient temperature, the calcium carbonate saturation state of
ambient seawater is a crucial variable determining calcification rates.
The calcium carbonate saturation state of seawater is defined as the
product of the calcium ion (\(Ca^{2+}\)) and carbonate ion
(\(CO_3^{2+}\)) concentrations in the ocean divided by the solubility
product (\(K_{sp}\)). A reduction in concentration of either calcium or
carbonate ions thus reduces the seawater saturation state impairing
shell formation (Gattuso et al. 1998; Schneider and Erez 2006) and
potentially increasing shell dissolution (Sanders et al. 2021). Lowered
seawater salinity is generally associated with lower calcium ion
concentrations, whereas decreasing seawater pH is accompanied by a shift
in the carbonate system which reduces the concentration of carbonate
ions.

We found that median seawater salinities varied significantly between
sampling sites by up to 2.6 psu from 30.1 psu at Zwarte Polder to 32.7
psu at Oostende which is \(\sim85\) \% more variable than predicted mean
salinity changes on the Belgian coast over the last century (Telesca et
al. 2021). This suggests that salinity effects on shell formation in
this study, if present, should be stronger on a spatial than a temporal
scale. However, we did not identitfy any significant trends between
salinities and shell layer thickness. This suggests that the
distribution of \emph{N. lapillus} salinity optimum is rather broad, so
that median salinity differences between sampling sites were not
sufficiently large to cause stress or reduce physiological performance
to the point where it affected shell production in the studied specimens
(Supplementary Figure 3). A zero-effect would also be expected
considering that salinities at either sampling site remained
sufficiently high to maintain saturation states \(\Omega\)
\textgreater{} 1, favoring calcium carbonate precipitation.

Contrary to the effect of salinity on calcium ion concentration,
decreasing seawater pH as in global OA predictions is accompanied by a
change in the seawater carbonate system, shifting the equilibrium away
from the carbonate ion towards carbonic acid reducing the calcium
carbonate saturation state of the seawater. Laboratory experiments often
find a thinning of shells as a response to OA (Fitzer et al. 2015;
Gaylord et al. 2011) although the opposite has also been reported
(Cross, Peck, and Harper 2015; Cross, Harper, and Peck 2019). Organism
responses to OA in nature appear to depend on a number of factors and
their interactions, making it difficult to draw general conclusions. To
make things more complicated organism responses to OA can take on
unexpected forms. For example, a long-term study comparing aragonite
percentage in \emph{Mytilus californianus} collected in the 1950s and
2010s reported a significant decrease in aragonite content (as a
percentage of whole shell \(CaCO_3\)) and an increase in calcite content
(Bullard et al. 2021). Although counterintuitive, calcite thickening has
been found to be an important response against OA induced shell
dissolution (Cross, Harper, and Peck 2019; Harper 2000; Telesca et al.
2021). Unfortunately, we could not directly assess the effect of pH on
shell formation due to a lack of a long-term pH record for the Southern
North Sea coast. Long-term pH trends of the greater North Sea however
appear to follow global OA trends (Bates et al. 2014). Assimilated data
by the International Council of the Exploration of the Seas (ICES) show
an average drop of 0.0035 ±0.0014 pH units per year from 1984 -- 2014
for much of the North Sea and the English Channel (Ostle et al. 2016).
In contrast, near shore carbonate system model studies (Borges and
Gypens 2010; Gypens, Borges, and Lancelot 2009) report increasing
seawater pH for the years 1951 to 1988, followed by overall stable,
albeit increasingly more variable, pH conditions (Desmit et al. 2020).
The authors attributed this to increasing riverine carbon and nutrient
loads since the 1960s causing extensive eutrophication of the Belgian
near shore. The input of e.g., riverine water with high nutrient loads
caused a local alkalinisation. This buffered ocean acidification effects
and kept mean annual saturation levels high (\(\Omega_{Ar}\)
\textgreater{} 2.5 and \(\Omega_{Ca}\) \textgreater{} 4) in the second
half of the last century (Borges and Gypens 2010), which provided
favourable conditions for \emph{N. lapillus} (Petraitis and Dudgeon
2020). In more recent years a decreasing trend in seawater pH was
observed due to a de-eutrophication of the Belgian near shore (Desmit et
al. 2020), which will likely impact shell formation in future
generations of \emph{N. lapillus}.

\hypertarget{food-abundance-a-possible-factor}{%
\subsubsection{Food abundance a possible
factor?}\label{food-abundance-a-possible-factor}}

In addition to buffering OA effects, coastal eutrophication may have had
an additional beneficial effect on the carnivorous \emph{N. lapillus}
for which there is only anecdotal evidence. Moderate eutrophication can
enhance the development of fouling communities (Page and Hubbard 1987;
Thomsen et al. 2013; Wołowicz et al. 2006), whereas high eutrophication
levels can be detrimental to local community structures and increase
densities of more tolerant species (Moran and Grant 1989, 1991;
Rastetter and Cooke 1979). As explained previously, shell formation
comes with costs in metabolic energy suggesting that food abundance can
play a limiting role in the rate of shell formation (Palmer 1983, 1992).
It is well established that the eutrophication on the Belgian near shore
led to increased primary production providing increasing nutrient
supplies for filter feeders such as barnacles and Mytilus spp., both
preferred prey of \emph{N. lapillus} (Desmit et al. 2020; Gypens,
Borges, and Lancelot 2009; Telesca et al. 2021). In addition, the
introduction of the invasive barnacle species \emph{Austrominius
modestus} (Darwin, 1954) from New Zealand and southern Australia (Harms,
1998) and its rapid spreading on the Southern North Sea coast provided
extensive foraging grounds for \emph{N. lapillus}. Since its first
documented sighting on the Dutch coast in 1946 (Boschma 1948), \emph{A.
modestus} spread to most parts of the Dutch coast in only six years
(Wolff 2005). On the Belgian coast, \emph{A. modestus} has been reported
since 1950 (Leloup and Lefevere 1952), but it may have already been
present there earlier. \emph{Austrominius modestus} is now the dominant
barnacle species of the Belgian intertidal fauna (Kerckhof 2002),
densely covering available hard substrata as well as the shells of other
organisms (Supplementary Figure 6). Today, both \emph{A. modestus} and
Mytilus spp. are highly abundant on the Belgian and Dutch coasts,
providing a wealth of prey for \emph{N. lapillus} (Supplementary Figure
6; D.M. personal observation). Considering the persisting
de-eutrophication trend of the Belgian near shore since the 1980s this
effect could be expected to be similar, or even larger, in the latter
half of the last century suggesting that most specimens analysed here
must have obtained enough energy to cover the metabolic costs required
for shell production.

To summarize, observed shell layer thickening in archival \emph{N.
lapillus} from the Belgian and Dutch coasts appears to result from the
interaction of multiple abiotic and biotic conditions that provided
favourable conditions for shell formation throughout the last 130 years.
Increasing annual temperatures and extended annual calcification windows
provided better kinetic conditions for primary shell production and
longer annual periods for secondary shell layer thickening. In addition,
anthropogenically induced eutrophication mitigated OA effects
maintaining high carbonate saturation states at the Belgian coast.
Anecdotal evidence points towards a possible influence of enhanced prey
abundance on shell formation rates due to the eutrophication and the
rapid spreading of the alien barnacle species \emph{A. modestus} on the
Belgian coast since the 1950s. Our findings suggest that although OA and
climate change threaten global ecosystems, their spatial proliferation
can be patchy creating niches with environmental conditions favourable
for marine calcifiers like \emph{N. lapillus}. This gives hope that
within these niches marine calcifiers may survive predicted future OA
scenarios longer than expected, extending the point of no return into
the future.

\hypertarget{data-availability}{%
\section{Data availability}\label{data-availability}}

All data and code used in this study are made publicly available on the
following git repository:
\href{https://github.com/dm807cam/nucella2021belgium}{dm807cam/nucella2021belgium}.

\hypertarget{references}{%
\section*{References}\label{references}}
\addcontentsline{toc}{section}{References}

\hypertarget{refs}{}
\begin{CSLReferences}{1}{0}
\leavevmode\vadjust pre{\hypertarget{ref-Angilletta2013-ie}{}}%
Angilletta, Michael J, Jr, Maximilian H Zelic, Gregory J Adrian, Alex M
Hurliman, and Colton D Smith. 2013. {``Heat Tolerance During Embryonic
Development Has Not Diverged Among Populations of a Widespread Species
(Sceloporus Undulatus).''} \emph{Conserv Physiol} 1 (1): cot018.

\leavevmode\vadjust pre{\hypertarget{ref-Appleton1988-zb}{}}%
Appleton, R D, and A R Palmer. 1988. {``Water-Borne Stimuli Released by
Predatory Crabs and Damaged Prey Induce More Predator-Resistant Shells
in a Marine Gastropod.''} \emph{Proc. Natl. Acad. Sci. U. S. A.} 85
(12): 4387--91.

\leavevmode\vadjust pre{\hypertarget{ref-Ashton2017-cm}{}}%
Ashton, Gail V, Simon A Morley, David K A Barnes, Melody S Clark, and
Lloyd S Peck. 2017. {``Warming by 1°c Drives Species and Assemblage
Level Responses in Antarctica's Marine Shallows.''} \emph{Curr. Biol.}
27 (17): 2698--2705.e3.

\leavevmode\vadjust pre{\hypertarget{ref-Avery2006-zf}{}}%
Avery, R, and R J Etter. 2006. {``Microstructural Differences in the
Reinforcement of a Gastropod Shell Against Predation.''} \emph{Mar.
Ecol. Prog. Ser.} 323 (October): 159--70.

\leavevmode\vadjust pre{\hypertarget{ref-Barclay2019-xu}{}}%
Barclay, K M, B Gaylord, B M Jellison, P Shukla, E Sanford, and L R
Leighton. 2019. {``Variation in the Effects of Ocean Acidification on
Shell Growth and Strength in Two Intertidal Gastropods.''} \emph{Mar.
Ecol. Prog. Ser.} 626 (September): 109--21.

\leavevmode\vadjust pre{\hypertarget{ref-Barnes2021-ju}{}}%
Barnes, David K A, Gail V Ashton, Simon A Morley, and Lloyd S Peck.
2021. {``1 °c Warming Increases Spatial Competition Frequency and
Complexity in Antarctic Marine Macrofauna.''} \emph{Commun Biol} 4 (1):
208.

\leavevmode\vadjust pre{\hypertarget{ref-Bates2014-rh}{}}%
Bates, Nicholas R, Yrene M Astor, Matthew J Church, Kim Currie, John E
Dore, Melchor González-Dávila, Laura Lorenzoni, Frank Muller-Karger, Jon
Olafsson, and J Magdalena Santana-Casiano. 2014. {``A {Time-Series} View
of Changing Surface Ocean Chemistry Due to Ocean Uptake of Anthropogenic
{CO\_2} and Ocean Acidification.''} \emph{Oceanography} 27 (1): 126--41.

\leavevmode\vadjust pre{\hypertarget{ref-Bernhardt2018-fp}{}}%
Bernhardt, Joey R, Jennifer M Sunday, Patrick L Thompson, and Mary I
O'Connor. 2018. {``Nonlinear Averaging of Thermal Experience Predicts
Population Growth Rates in a Thermally Variable Environment.''}
\emph{Proc. Biol. Sci.} 285 (1886).

\leavevmode\vadjust pre{\hypertarget{ref-Berry2009-ql}{}}%
Berry, R J, and J H Crothers. 2009. {``Visible Variation in the
Dog-Whelk, Nucella Lapillus.''} \emph{J. Zool.} 174 (1): 123--48.

\leavevmode\vadjust pre{\hypertarget{ref-Bonhomme2014-jj}{}}%
Bonhomme, Vincent, Sandrine Picq, Cedric Gaucherel, and Julien Claude.
2014. {``Momocs: Outline Analysis Using {R}.''} \emph{Journal of
Statistical Software}.

\leavevmode\vadjust pre{\hypertarget{ref-Borges2010-jq}{}}%
Borges, Alberto V, and Nathalie Gypens. 2010. {``Carbonate Chemistry in
the Coastal Zone Responds More Strongly to Eutrophication Than Ocean
Acidification.''} \emph{Limnol. Oceanogr.} 55 (1): 346--53.

\leavevmode\vadjust pre{\hypertarget{ref-Boschma1948-fd}{}}%
Boschma, H. 1948. {``Elminius Modestus in the Netherlands.''}
\emph{Nature} 161 (4089): 403--4.

\leavevmode\vadjust pre{\hypertarget{ref-Brion2008-na}{}}%
Brion, Natacha, Siegrid Jans, Lei Chou, and Véronique Rousseau. 2008.
{``Nutrient Loads to the Belgian Coastal Zone.''} Edited by Rousseau,
V., Lancelot, C. \& Cox, D. \emph{Current Status of Eutrophication in
the Belgian Coastal Zone}, 17--43.

\leavevmode\vadjust pre{\hypertarget{ref-Bullard2021-lk}{}}%
Bullard, Elizabeth M, Ivan Torres, Tianqi Ren, Olivia A Graeve, and
Kaustuv Roy. 2021. {``Shell Mineralogy of a Foundational Marine Species,
Mytilus Californianus, over Half a Century in a Changing Ocean.''}
\emph{Proc. Natl. Acad. Sci. U. S. A.} 118 (3).

\leavevmode\vadjust pre{\hypertarget{ref-Burrows1989-gi}{}}%
Burrows, Michael T, and Roger N Hughes. 1989. {``Natural Foraging of the
Dogwhelk, Nucell Lapillus (Linnaeus); the Weather and Whether to
Feed.''} \emph{J. Molluscan Stud.} 55 (2): 285--95.

\leavevmode\vadjust pre{\hypertarget{ref-Burton1987-vn}{}}%
Burton, Elizabeth A, and Lynn M Walter. 1987. {``Relative Precipitation
Rates of Aragonite and Mg Calcite from Seawater: Temperature or
Carbonate Ion Control?''} \emph{Geology} 15 (2): 111--14.

\leavevmode\vadjust pre{\hypertarget{ref-Byrne2011-ql}{}}%
Byrne, Maria. 2011. {``Impact of Ocean Warming and Ocean Acidification
on Marine Invertebrate Life History Stages.''} In \emph{Oceanography and
Marine Biology}. Oceanography and Marine Biology: An Annual Review. CRC
Press.

\leavevmode\vadjust pre{\hypertarget{ref-Byrne2013-om}{}}%
Byrne, Maria, and Rachel Przeslawski. 2013. {``Multistressor Impacts of
Warming and Acidification of the Ocean on Marine Invertebrates' Life
Histories.''} \emph{Integr. Comp. Biol.} 53 (4): 582--96.

\leavevmode\vadjust pre{\hypertarget{ref-Clark2021-hy}{}}%
Clark, Melody S, Lloyd S Peck, and Jakob Thyrring. 2021. {``Resilience
in Greenland Intertidal Mytilus: The Hidden Stress Defense.''}
\emph{Sci. Total Environ.} 767 (May): 144366.

\leavevmode\vadjust pre{\hypertarget{ref-Clarke1993-lx}{}}%
Clarke, Andrew. 1993. {``Seasonal Acclimatization and Latitudinal
Compensation in Metabolism: Do They Exist?''} \emph{Funct. Ecol.} 7 (2):
139--49.

\leavevmode\vadjust pre{\hypertarget{ref-Clarke2020-fh}{}}%
Clarke, Andrew, and Jennifer C Beaumont. 2020. {``An Extreme Marine
Environment: A 14-Month Record of Temperature in a Polar Tidepool.''}
\emph{Polar Biol.} 43 (12): 2021--30.

\leavevmode\vadjust pre{\hypertarget{ref-Cornwall2020-ub}{}}%
Cornwall, C E, S Comeau, T M DeCarlo, E Larcombe, B Moore, K Giltrow, F
Puerzer, Q D'Alexis, and M T McCulloch. 2020. {``A Coralline Alga Gains
Tolerance to Ocean Acidification over Multiple Generations of
Exposure.''} \emph{Nat. Clim. Chang.} 10 (2): 143--46.

\leavevmode\vadjust pre{\hypertarget{ref-Cowell1970-jq}{}}%
Cowell, E B, and J H Crothers. 1970. {``On the Occurrence of Multiple
Rows of {`Teeth'} in the Shell of the {Dog-Whelk} Nucella Lapillus.''}
\emph{J. Mar. Biol. Assoc. U. K.} 50 (4): 1101--11.

\leavevmode\vadjust pre{\hypertarget{ref-Cross2018-bw}{}}%
Cross, Emma L, Elizabeth M Harper, and Lloyd S Peck. 2018. {``A 120-Year
Record of Resilience to Environmental Change in Brachiopods.''}
\emph{Glob. Chang. Biol.} 24 (6): 2262--71.

\leavevmode\vadjust pre{\hypertarget{ref-Cross2019-kd}{}}%
---------. 2019. {``Thicker Shells Compensate Extensive Dissolution in
Brachiopods Under Future Ocean Acidification.''} \emph{Environ. Sci.
Technol.} 53 (9): 5016--26.

\leavevmode\vadjust pre{\hypertarget{ref-Cross2015-np}{}}%
Cross, Emma L, Lloyd S Peck, and Elizabeth M Harper. 2015. {``Ocean
Acidification Does Not Impact Shell Growth or Repair of the Antarctic
Brachiopod Liothyrella Uva (Broderip, 1833).''} \emph{J. Exp. Mar. Bio.
Ecol.} 462 (January): 29--35.

\leavevmode\vadjust pre{\hypertarget{ref-Cross2015-rr}{}}%
Cross, Emma L, Lloyd S Peck, Miles D Lamare, and Elizabeth M Harper.
2015. {``No Ocean Acidification Effects on Shell Growth and Repair in
the New Zealand Brachiopod Calloria Inconspicua (Sowerby, 1846).''}
\emph{ICES J. Mar. Sci.} 73 (3): 920--26.

\leavevmode\vadjust pre{\hypertarget{ref-Crothers1985-cq}{}}%
Crothers, J H. 1985. {``Two Different Patterns of Shell-Shape Variation
in the Dog-Whelk Nucella Lapillus (l.).''} \emph{Biol. J. Linn. Soc.
Lond.} 25 (4): 339--53.

\leavevmode\vadjust pre{\hypertarget{ref-Currey1982-mu}{}}%
Currey, J D, and Roger N Hughes. 1982. {``Strength of the Dogwhelk
Nucella Lapillus and the Winkle Littorina Littorea from Different
Habitats.''} \emph{J. Anim. Ecol.} 51 (1): 47--56.

\leavevmode\vadjust pre{\hypertarget{ref-De_Blauwe2012-gg}{}}%
De Blauwe, H, and Cdu D'acoz. 2012. {``Voortplantende Populatie van de
Purperslak (Nucella Lapillus) in Belgi{ë} Na Meer Dan 30 Jaar
Afwezigheid (Mollusca, Gastropoda, Muricidae).''} \emph{Vliz.be} 32 (4):
127--31.

\leavevmode\vadjust pre{\hypertarget{ref-Desmit2020-ay}{}}%
Desmit, Xavier, Anja Nohe, Alberto Vieira Borges, Theo Prins, Karien De
Cauwer, Ruth Lagring, Dimitry Van der Zande, and Koen Sabbe. 2020.
{``Changes in Chlorophyll Concentration and Phenology in the North Sea
in Relation to de‐eutrophication and Sea Surface Warming.''}
\emph{Limnol. Oceanogr.} 65 (4): 828--47.

\leavevmode\vadjust pre{\hypertarget{ref-Donelson2018-py}{}}%
Donelson, Jennifer M, Santiago Salinas, Philip L Munday, and Lisa N S
Shama. 2018. {``Transgenerational Plasticity and Climate Change
Experiments: Where Do We Go from Here?''} \emph{Glob. Chang. Biol.} 24
(1): 13--34.

\leavevmode\vadjust pre{\hypertarget{ref-Doney2009-wq}{}}%
Doney, Scott C, Victoria J Fabry, Richard A Feely, and Joan A Kleypas.
2009. {``Ocean Acidification: The Other {Co2} Problem.''} \emph{Ann.
Rev. Mar. Sci.} 1: 169--92.

\leavevmode\vadjust pre{\hypertarget{ref-Doroudi1999-ao}{}}%
Doroudi, M S, P C Southgate, and R J Mayer. 1999. {``The Combined
Effects of Temperature and Salinity on Embryos and Larvae of the
Black‐lip Pearl Oyster, Pinctada Margaritifera (l.).''} \emph{Aquac.
Res.} 30 (4): 271--77.

\leavevmode\vadjust pre{\hypertarget{ref-Easterling2000-cw}{}}%
Easterling, D R, G A Meehl, C Parmesan, S A Changnon, T R Karl, and L O
Mearns. 2000. {``Climate Extremes: Observations, Modeling, and
Impacts.''} \emph{Science} 289 (5487): 2068--74.

\leavevmode\vadjust pre{\hypertarget{ref-Fay2013-vi}{}}%
Fay, A R, and G A McKinley. 2013. {``Global Trends in Surface
{oceanpCO2from} in Situ Data.''} \emph{Global Biogeochem. Cycles} 27
(2): 541--57.

\leavevmode\vadjust pre{\hypertarget{ref-Fisher2009-dv}{}}%
Fisher, Jonathan A D, Erika C Rhile, Harrison Liu, and Peter S
Petraitis. 2009. {``An Intertidal Snail Shows a Dramatic Size Increase
over the Past Century.''} \emph{Proc. Natl. Acad. Sci. U. S. A.} 106
(13): 5209--12.

\leavevmode\vadjust pre{\hypertarget{ref-Fitzer2014-mg}{}}%
Fitzer, Susan C, Maggie Cusack, Vernon R Phoenix, and Nicholas A
Kamenos. 2014. {``Ocean Acidification Reduces the Crystallographic
Control in Juvenile Mussel Shells.''} \emph{J. Struct. Biol.} 188 (1):
39--45.

\leavevmode\vadjust pre{\hypertarget{ref-Fitzer2015-nr}{}}%
Fitzer, Susan C, Liberty Vittert, Adrian Bowman, Nicholas A Kamenos,
Vernon R Phoenix, and Maggie Cusack. 2015. {``Ocean Acidification and
Temperature Increase Impact Mussel Shell Shape and Thickness:
Problematic for Protection?''} \emph{Ecol. Evol.} 5 (21): 4875--84.

\leavevmode\vadjust pre{\hypertarget{ref-Galante-Oliveira2011-ux}{}}%
Galante-Oliveira, Susana, Raquel Marçal, Mário Pacheco, and Carlos M
Barroso. 2011. {``Nucella Lapillus Ecotypes at the Southern
Distributional Limit in Europe: Variation in Shell Morphology Is Not
Correlated with Chromosome Counts on the Portuguese Atlantic Coast.''}
\emph{J. Molluscan Stud.} 78 (1): 147--50.

\leavevmode\vadjust pre{\hypertarget{ref-Galante-Oliveira2010-ha}{}}%
Galante-Oliveira, Susana, Isabel Oliveira, José António Santos, Maria de
Lourdes Pereira, Mário Pacheco, and Carlos M Barroso. 2010. {``Factors
Affecting {RPSI} in Imposex Monitoring Studies Using Nucella Lapillus
(l.) As Bioindicator.''} \emph{J. Environ. Monit.} 12 (5): 1055--63.

\leavevmode\vadjust pre{\hypertarget{ref-Gattuso1998-sg}{}}%
Gattuso, J-P, M Frankignoulle, I Bourge, S Romaine, and R W Buddemeier.
1998. {``Effect of Calcium Carbonate Saturation of Seawater on Coral
Calcification.''} \emph{Glob. Planet. Change} 18 (1): 37--46.

\leavevmode\vadjust pre{\hypertarget{ref-Gaylord2011-aj}{}}%
Gaylord, Brian, Tessa M Hill, Eric Sanford, Elizabeth A Lenz, Lisa A
Jacobs, Kirk N Sato, Ann D Russell, and Annaliese Hettinger. 2011.
{``Functional Impacts of Ocean Acidification in an Ecologically Critical
Foundation Species.''} \emph{J. Exp. Biol.} 214 (Pt 15): 2586--94.

\leavevmode\vadjust pre{\hypertarget{ref-Gazeau2013-cs}{}}%
Gazeau, Frédéric, Laura M Parker, Steeve Comeau, Jean-Pierre Gattuso,
Wayne A O'Connor, Sophie Martin, Hans-Otto Pörtner, and Pauline M Ross.
2013. {``Impacts of Ocean Acidification on Marine Shelled Molluscs.''}
\emph{Mar. Biol.} 160 (8): 2207--45.

\leavevmode\vadjust pre{\hypertarget{ref-Giardina1977-jo}{}}%
Giardina, Charles R, and Frank P Kuhl. 1977. {``Accuracy of Curve
Approximation by Harmonically Related Vectors with Elliptical Loci.''}
\emph{Computer Graphics and Image Processing} 6 (3): 277--85.

\leavevmode\vadjust pre{\hypertarget{ref-Gibbin2017-kh}{}}%
Gibbin, Emma M, Gloria Massamba N'Siala, Leela J Chakravarti, Michael D
Jarrold, and Piero Calosi. 2017. {``The Evolution of Phenotypic
Plasticity Under Global Change.''} \emph{Sci. Rep.} 7 (1): 17253.

\leavevmode\vadjust pre{\hypertarget{ref-Gibbs1999-qr}{}}%
Gibbs, Peter E. 1999. {``Biological Effects of Contaminants: Use of
Imposex in the Dogwhelk (Nucella Lapillus) as a Bioindicator of
Tributyltin Pollution.''} International Council for the Exploration of
the Sea (ICES).

\leavevmode\vadjust pre{\hypertarget{ref-Gibbs1991-rr}{}}%
Gibbs, Peter E, G W Bryan, and P L Pascoe. 1991. {``{TBT-induced}
Imposex in the Dogwhelk, Nucella Lapillus: Geographical Uniformity of
the Response and Effects.''} \emph{Mar. Environ. Res.} 32 (1): 79--87.

\leavevmode\vadjust pre{\hypertarget{ref-Gypens2009-ok}{}}%
Gypens, N, A V Borges, and C Lancelot. 2009. {``Effect of Eutrophication
on Air-Sea {CO2fluxes} in the Coastal Southern North Sea: A Model Study
of the Past 50 Years.''} \emph{Glob. Chang. Biol.} 15 (4): 1040--56.

\leavevmode\vadjust pre{\hypertarget{ref-Harley2017-so}{}}%
Harley, Christopher D G, Sean D Connell, Zoë A Doubleday, Brendan
Kelaher, Bayden D Russell, Gianluca Sarà, and Brian Helmuth. 2017.
{``Conceptualizing Ecosystem Tipping Points Within a Physiological
Framework.''} \emph{Ecol. Evol.} 7 (15): 6035--45.

\leavevmode\vadjust pre{\hypertarget{ref-Harper2000-xx}{}}%
Harper, E M. 2000. {``Are Calcitic Layers an Effective Adaptation
Against Shell Dissolution in the Bivalvia?''} \emph{J. Zool.} 251 (2):
179--86.

\leavevmode\vadjust pre{\hypertarget{ref-Hofmann2010-uq}{}}%
Hofmann, Gretchen E, James P Barry, Peter J Edmunds, Ruth D Gates, David
A Hutchins, Terrie Klinger, and Mary A Sewell. 2010. {``The Effect of
Ocean Acidification on Calcifying Organisms in Marine Ecosystems: An
{Organism-to-Ecosystem} Perspective.''} \emph{Annu. Rev. Ecol. Evol.
Syst.} 41 (1): 127--47.

\leavevmode\vadjust pre{\hypertarget{ref-Hofmann1995-yv}{}}%
Hofmann, Gretchen E, and G Somero. 1995. {``Evidence for Protein Damage
at Environmental Temperatures: Seasonal Changes in Levels of Ubiquitin
Conjugates and Hsp70 in the Intertidal Mussel Mytilus Trossulus.''}
\emph{J. Exp. Biol.} 198 (Pt 7): 1509--18.

\leavevmode\vadjust pre{\hypertarget{ref-Hughes1994-qn}{}}%
Hughes, Roger N, and M T Burrows. 1994. {``An Interdisciplinary Approach
to the Study of Foraging Behaviour in the Predatory Gastropod, Nucella
Lapillus (l.).''} \emph{Ethol. Ecol. Evol.} 6 (1): 75--85.

\leavevmode\vadjust pre{\hypertarget{ref-Hughes1979-qc}{}}%
Hughes, Roger N, and R W Elner. 1979. {``Tactics of a Predator, Carcinus
Maenas, and Morphological Responses of the Prey, Nucella Lapillus.''}
\emph{J. Anim. Ecol.} 48 (1): 65--78.

\leavevmode\vadjust pre{\hypertarget{ref-Ices2020-be}{}}%
ICES. 2020. {``Dataset on Ocean Hydrography.''} ICES, Copenhagen.

\leavevmode\vadjust pre{\hypertarget{ref-Joubert2014-vf}{}}%
Joubert, Caroline, Clémentine Linard, Gilles Le Moullac, Claude Soyez,
Denis Saulnier, Vaihiti Teaniniuraitemoana, Chin Long Ky, and Yannick
Gueguen. 2014. {``Temperature and Food Influence Shell Growth and Mantle
Gene Expression of Shell Matrix Proteins in the Pearl Oyster Pinctada
Margaritifera.''} \emph{PLoS One} 9 (8): e103944.

\leavevmode\vadjust pre{\hypertarget{ref-Kerckhof1988-pz}{}}%
Kerckhof, F. 1988. {``Over Het Verdwijnen van de Purperslak Nucella
Lapillus (Linnaeus 1758). Langs Onze Kust.''} \emph{De Strandvlo}, no.
8(2): 82--85.

\leavevmode\vadjust pre{\hypertarget{ref-Kerckhof2002-rt}{}}%
---------. 2002. {``Barnacles (Cirripedia, Balanomorpha) in Belgian
Waters, an Overview of the Species and Recent Evolutions, with Emphasis
on Exotic Species.''} \emph{Bull. Inst. R. Sci. Nat. Belgique
Biol./Bull. K. Belgisch Inst. Nat. Biol.} 72: 93--104.

\leavevmode\vadjust pre{\hypertarget{ref-Kroeker2020-uq}{}}%
Kroeker, Kristy J, Lauren E Bell, Emily M Donham, Umihiko Hoshijima,
Sarah Lummis, Jason A Toy, and Ellen Willis-Norton. 2020. {``Ecological
Change in Dynamic Environments: Accounting for Temporal Environmental
Variability in Studies of Ocean Change Biology.''} \emph{Glob. Chang.
Biol.} 26 (1): 54--67.

\leavevmode\vadjust pre{\hypertarget{ref-Kroeker2013-nr}{}}%
Kroeker, Kristy J, Rebecca L Kordas, Ryan Crim, Iris E Hendriks, Laura
Ramajo, Gerald S Singh, Carlos M Duarte, and Jean-Pierre Gattuso. 2013.
{``Impacts of Ocean Acidification on Marine Organisms: Quantifying
Sensitivities and Interaction with Warming.''} \emph{Glob. Chang. Biol.}
19 (6): 1884--96.

\leavevmode\vadjust pre{\hypertarget{ref-Kroeker2016-ma}{}}%
Kroeker, Kristy J, Eric Sanford, Jeremy M Rose, Carol A Blanchette,
Francis Chan, Francisco P Chavez, Brian Gaylord, et al. 2016.
{``Interacting Environmental Mosaics Drive Geographic Variation in
Mussel Performance and Predation Vulnerability.''} \emph{Ecol. Lett.} 19
(7): 771--79.

\leavevmode\vadjust pre{\hypertarget{ref-Kuhl1982-jb}{}}%
Kuhl, Frank P, and Charles R Giardina. 1982. {``Elliptic Fourier
Features of a Closed Contour.''} \emph{Computer Graphics and Image
Processing} 18 (3): 236--58.

\leavevmode\vadjust pre{\hypertarget{ref-Laing2000-wd}{}}%
Laing, Ian. 2000. {``Effect of Temperature and Ration on Growth and
Condition of King Scallop (Pecten Maximus) Spat.''} \emph{Aquaculture}
183 (3): 325--34.

\leavevmode\vadjust pre{\hypertarget{ref-Leloup1952-ez}{}}%
Leloup, E, and S Lefevere. 1952. {``Sur La Pr{é}sence Dans Les Eaux de
La c{ô}te Belge Du Cirrip{è}de.''} \emph{Elminius Modestus}, 1--12.

\leavevmode\vadjust pre{\hypertarget{ref-Mackenzie2002-gg}{}}%
Mackenzie, Fred T, Leah May Ver, and Abraham Lerman. 2002.
{``Century-Scale Nitrogen and Phosphorus Controls of the Carbon
Cycle.''} \emph{Chem. Geol.} 190 (1): 13--32.

\leavevmode\vadjust pre{\hypertarget{ref-Marra2011-ko}{}}%
Marra, Giampiero, and Simon N Wood. 2011. {``Practical Variable
Selection for Generalized Additive Models.''} \emph{Comput. Stat. Data
Anal.} 55 (7): 2372--87.

\leavevmode\vadjust pre{\hypertarget{ref-Mayk2020-ko}{}}%
Mayk, Dennis. 2020. {``Transitional Spherulitic Layer in the Muricid
Nucella Lapillus.''} \emph{J. Molluscan Stud.} 87 (1).

\leavevmode\vadjust pre{\hypertarget{ref-Meehl2004-ur}{}}%
Meehl, Gerald A, and Claudia Tebaldi. 2004. {``More Intense, More
Frequent, and Longer Lasting Heat Waves in the 21st Century.''}
\emph{Science} 305 (5686): 994--97.

\leavevmode\vadjust pre{\hypertarget{ref-Moran1989-or}{}}%
Moran, P J, and T R Grant. 1989. {``The Effects of Industrial Pollution
on the Development and Succession of Marine Fouling Communities i.
Analysis of Species Richness and Frequency Data.''} \emph{Mar. Ecol.} 10
(3): 231--46.

\leavevmode\vadjust pre{\hypertarget{ref-Moran1991-kx}{}}%
---------. 1991. {``Transference of Marine Fouling Communities Between
Polluted and Unpolluted Sites: Impact on Structure.''} \emph{Environ.
Pollut.} 72 (2): 89--102.

\leavevmode\vadjust pre{\hypertarget{ref-National_Centers_for_Environmental_InformationNESDISNOAAUS_Department_of_Commerce2016-ey}{}}%
National Centers for Environmental Information/NESDIS/NOAA/U.S.
Department of Commerce, Research Data Archive/Computational and
Information Systems Laboratory/National Center for Atmospheric
Research/University Corporation for Atmospheric Research, Earth System
Research Laboratory/NOAA/U.S. Department of Commerce, Cooperative
Institute for Research in Environmental Sciences/University of Colorado,
National Oceanography Centre/Natural Environment Research Council/United
Kingdom, Met Office/Ministry of Defence/United Kingdom, Deutscher
Wetterdienst (German Meteorological Service)/Germany, Department of
Atmospheric Science/University of Washington, and Center for
Ocean-Atmospheric Prediction Studies/Florida State University. 2016.
{``International Comprehensive {Ocean-Atmosphere} Data Set ({ICOADS})
Release 3, Monthly Summaries.''} UCAR/NCAR - Research Data Archive.

\leavevmode\vadjust pre{\hypertarget{ref-Nehls1993-lz}{}}%
Nehls, Georg, and Martin Thiel. 1993. {``Large-Scale Distribution
Patterns of the Mussel Mytilus Edulis in the Wadden Sea of
{Schleswig-Holstein}: Do Storms Structure the Ecosystem?''} \emph{Neth.
J. Sea Res.} 31 (2): 181--87.

\leavevmode\vadjust pre{\hypertarget{ref-Oehlmann1998-zg}{}}%
Oehlmann, J, B Bauer, D Minchin, U Schulte-Oehlmann, P Fioroni, and B
Markert. 1998. {``Imposex in Nucella Lapillus and Intersex in Littorina
Littorea: Interspecific Comparison of Two {TBT-induced} Effects and
Their Geographical Uniformity.''} In \emph{Aspects of Littorinid
Biology}, edited by Ruth M O'Riordan, Gavin M Burnell, Mark S Davies,
and Neil F Ramsay, 199--213. Dordrecht: Springer Netherlands.

\leavevmode\vadjust pre{\hypertarget{ref-Ostle2016-kg}{}}%
Ostle, C, P Willliamson, Y Artioli, D C E Bakker, S N R Birchenough, C E
Davis, S Dye, et al. 2016. {``Carbon Dioxide and Ocean Acidification
Observations in {UK} Waters. Synthesis Report with a Focus on
2010--2015,''} June, 44.

\leavevmode\vadjust pre{\hypertarget{ref-Page1987-tb}{}}%
Page, H M, and D M Hubbard. 1987. {``Temporal and Spatial Patterns of
Growth in Mussels Mytilus Edulis on an Offshore Platform: Relationships
to Water Temperature and Food Availability.''} \emph{J. Exp. Mar. Bio.
Ecol.} 111 (2): 159--79.

\leavevmode\vadjust pre{\hypertarget{ref-Palmer1990-qn}{}}%
Palmer. 1990. {``Effect of Crab Effluent and Scent of Damaged
Conspecifics on Feeding, Growth, and Shell Morphology of the Atlantic
Dogwhelk Nucella Lapillus (l.).''} \emph{Hydrobiologia} 193 (1):
155--82.

\leavevmode\vadjust pre{\hypertarget{ref-Palmer1983-nq}{}}%
Palmer, A R. 1983. {``Relative Cost of Producing Skeletal Organic Matrix
Versus Calcification: Evidence from Marine Gastropods.''} \emph{Mar.
Biol.} 75 (2): 287--92.

\leavevmode\vadjust pre{\hypertarget{ref-Palmer1992-uz}{}}%
---------. 1992. {``Calcification in Marine Molluscs: How Costly Is
It?''} \emph{Proceedings of the National Academy of the United States of
America}, no. 89: 1379--82.

\leavevmode\vadjust pre{\hypertarget{ref-Parker2013-zx}{}}%
Parker, Laura M, Pauline M Ross, Wayne A O'Connor, Hans O Pörtner,
Elliot Scanes, and John M Wright. 2013. {``Predicting the Response of
Molluscs to the Impact of Ocean Acidification.''} \emph{Biology} 2 (2):
651--92.

\leavevmode\vadjust pre{\hypertarget{ref-Pascoal2012-od}{}}%
Pascoal, Sonia, Gary Carvalho, Simon Creer, Sonia Mendo, and Roger N
Hughes. 2012. {``Plastic and Heritable Variation in Shell Thickness of
the Intertidal Gastropod Nucella Lapillus Associated with Risks of Crab
Predation and Wave Action, and Sexual Maturation.''} Edited by Stephen J
Martin. \emph{PLoS One} 7 (12): e52134.

\leavevmode\vadjust pre{\hypertarget{ref-Peck2011-ua}{}}%
Peck, Lloyd S. 2011. {``Organisms and Responses to Environmental
Change.''} \emph{Mar. Genomics} 4 (4): 237--43.

\leavevmode\vadjust pre{\hypertarget{ref-Petraitis2020-gt}{}}%
Petraitis, Peter S, and S R Dudgeon. 2020. {``Declines over the Last Two
Decades of Five Intertidal Invertebrate Species in the Western North
Atlantic.''} \emph{Commun Biol} 3 (1): 591.

\leavevmode\vadjust pre{\hypertarget{ref-Przeslawski2015-ph}{}}%
Przeslawski, Rachel, Maria Byrne, and Camille Mellin. 2015. {``A Review
and Meta-Analysis of the Effects of Multiple Abiotic Stressors on Marine
Embryos and Larvae.''} \emph{Glob. Chang. Biol.} 21 (6): 2122--40.

\leavevmode\vadjust pre{\hypertarget{ref-R_Core_Team2020-ph}{}}%
R Core Team. 2020. {``R: A Language and Environment for Statistical
Computing.''} Vienna, Austria: R Foundation for Statistical Computing.

\leavevmode\vadjust pre{\hypertarget{ref-Rahmstorf2011-rh}{}}%
Rahmstorf, Stefan, and Dim Coumou. 2011. {``Increase of Extreme Events
in a Warming World.''} \emph{Proc. Natl. Acad. Sci. U. S. A.} 108 (44):
17905--9.

\leavevmode\vadjust pre{\hypertarget{ref-Rastetter1979-jj}{}}%
Rastetter, E B, and W J Cooke. 1979. {``Responses of Marine Fouling
Communities to Sewage Abatement in Kaneohe Bay, Oahu, Hawaii.''}
\emph{Mar. Biol.} 53 (3): 271--80.

\leavevmode\vadjust pre{\hypertarget{ref-Roberts1997-hi}{}}%
Roberts, D A, G E Hofmann, and G N Somero. 1997. {``{Heat-Shock} Protein
Expression in Mytilus Californianus: Acclimatization (Seasonal and
{Tidal-Height} Comparisons) and Acclimation Effects.''} \emph{Biol.
Bull.} 192 (2): 309--20.

\leavevmode\vadjust pre{\hypertarget{ref-Rummukainen2012-ej}{}}%
Rummukainen, Markku. 2012. {``Changes in Climate and Weather Extremes in
the 21st Century.''} \emph{Wiley Interdiscip. Rev. Clim. Change} 3 (2):
115--29.

\leavevmode\vadjust pre{\hypertarget{ref-Sanders2018-gr}{}}%
Sanders, Trystan, Lara Schmittmann, Jennifer C Nascimento-Schulze, and
Frank Melzner. 2018. {``High Calcification Costs Limit Mussel Growth at
Low Salinity.''} \emph{Frontiers in Marine Science} 5: 352.

\leavevmode\vadjust pre{\hypertarget{ref-Sanders2021-gf}{}}%
Sanders, Trystan, Jörn Thomsen, Jens Daniel Müller, Gregor Rehder, and
Frank Melzner. 2021. {``Decoupling Salinity and Carbonate Chemistry: Low
Calcium Ion Concentration Rather Than Salinity Limits Calcification in
Baltic Sea Mussels.''} \emph{Biogeosciences} 18 (8): 2573--90.

\leavevmode\vadjust pre{\hypertarget{ref-Schneider2006-nh}{}}%
Schneider, Kenneth, and Jonathan Erez. 2006. {``The Effect of Carbonate
Chemistry on Calcification and Photosynthesis in the Hermatypic Coral
Acropora Eurystoma.''} \emph{Limnol. Oceanogr.} 51 (3): 1284--93.

\leavevmode\vadjust pre{\hypertarget{ref-Somero2012-pq}{}}%
Somero, George N. 2012. {``The Physiology of Global Change: Linking
Patterns to Mechanisms.''} \emph{Ann. Rev. Mar. Sci.} 4: 39--61.

\leavevmode\vadjust pre{\hypertarget{ref-Suckling2015-hv}{}}%
Suckling, Coleen C, Melody S Clark, Joelle Richard, Simon A Morley,
Michael A S Thorne, Elizabeth M Harper, and Lloyd S Peck. 2015. {``Adult
Acclimation to Combined Temperature and pH Stressors Significantly
Enhances Reproductive Outcomes Compared to Short-Term Exposures.''}
\emph{J. Anim. Ecol.} 84 (3): 773--84.

\leavevmode\vadjust pre{\hypertarget{ref-Telesca2018-oo}{}}%
Telesca, Luca, Kati Michalek, Trystan Sanders, Lloyd S Peck, Jakob
Thyrring, and Elizabeth M Harper. 2018. {``Blue Mussel Shell Shape
Plasticity and Natural Environments: A Quantitative Approach.''}
\emph{Sci. Rep.} 8 (1): 2865.

\leavevmode\vadjust pre{\hypertarget{ref-Telesca2021-zm}{}}%
Telesca, Luca, Lloyd S Peck, Thierry Backeljau, Mario F Heinig, and
Elizabeth M Harper. 2021. {``A Century of Coping with Environmental and
Ecological Changes via Compensatory Biomineralization in Mussels.''}
\emph{Glob. Chang. Biol.} 27 (3): 624--39.

\leavevmode\vadjust pre{\hypertarget{ref-Telesca2019-ua}{}}%
Telesca, Luca, Lloyd S Peck, Trystan Sanders, Jakob Thyrring, Mikael K
Sejr, and Elizabeth M Harper. 2019. {``Biomineralization Plasticity and
Environmental Heterogeneity Predict Geographical Resilience Patterns of
Foundation Species to Future Change.''} \emph{Glob. Chang. Biol.} 25
(12): 4179--93.

\leavevmode\vadjust pre{\hypertarget{ref-Thomsen2013-pj}{}}%
Thomsen, Jörn, Isabel Casties, Christian Pansch, Arne Körtzinger, and
Frank Melzner. 2013. {``Food Availability Outweighs Ocean Acidification
Effects in Juvenile Mytilus Edulis: Laboratory and Field Experiments.''}
\emph{Glob. Chang. Biol.} 19 (4): 1017--27.

\leavevmode\vadjust pre{\hypertarget{ref-Thomsen2015-wi}{}}%
Thomsen, Jörn, K Haynert, K M Wegner, and F Melzner. 2015. {``Impact of
Seawater Carbonate Chemistry on the Calcification of Marine Bivalves.''}
\emph{Biogeosciences} 12 (14): 4209--20.

\leavevmode\vadjust pre{\hypertarget{ref-Thyrring2017-sd}{}}%
Thyrring, J, M E Blicher, J G Sørensen, S Wegeberg, and M K Sejr. 2017.
{``Rising Air Temperatures Will Increase Intertidal Mussel Abundance in
the Arctic.''} \emph{Mar. Ecol. Prog. Ser.} 584 (December): 91--104.

\leavevmode\vadjust pre{\hypertarget{ref-Vermeij1980-jt}{}}%
Vermeij, Geerat J, and John D Currey. 1980. {``Geographical Variation in
the Shell Strength of Thaidid Snail Shells.''} \emph{Biol. Bull.} 158
(3): 383--89.

\leavevmode\vadjust pre{\hypertarget{ref-Waldbusser2014-qn}{}}%
Waldbusser, George G, Burke Hales, Chris J Langdon, Brian A Haley, Paul
Schrader, Elizabeth L Brunner, Matthew W Gray, Cale A Miller, and Iria
Gimenez. 2014. {``Saturation-State Sensitivity of Marine Bivalve Larvae
to Ocean Acidification.''} \emph{Nat. Clim. Chang.} 5 (3): 273--80.

\leavevmode\vadjust pre{\hypertarget{ref-Waldbusser2015-ll}{}}%
Waldbusser, George G, Burke Hales, Chris J Langdon, Brian A Haley, Paul
Schrader, Elizabeth L Brunner, Matthew W Gray, Cale A Miller, Iria
Gimenez, and Greg Hutchinson. 2015. {``Ocean Acidification Has Multiple
Modes of Action on Bivalve Larvae.''} \emph{PLoS One} 10 (6): e0128376.

\leavevmode\vadjust pre{\hypertarget{ref-Watson2017-lv}{}}%
Watson, Sue-Ann, Simon A Morley, and Lloyd S Peck. 2017. {``Latitudinal
Trends in Shell Production Cost from the Tropics to the Poles.''}
\emph{Sci Adv} 3 (9): e1701362.

\leavevmode\vadjust pre{\hypertarget{ref-Watson2012-fy}{}}%
Watson, Sue-Ann, Lloyd S Peck, Paul A Tyler, Paul C Southgate, Koh Siang
Tan, Robert W Day, and Simon A Morley. 2012. {``Marine Invertebrate
Skeleton Size Varies with Latitude, Temperature and Carbonate
Saturation: Implications for Global Change and Ocean Acidification.''}
\emph{Glob. Chang. Biol.} 18 (10): 3026--38.

\leavevmode\vadjust pre{\hypertarget{ref-Wolff2005-ed}{}}%
Wolff, Willem Jan. 2005. {``Non-Indigenous Marine and Estuarine Species
in the Netherlands.''} \emph{Zoologische Mededelingen} 79: 1--116.

\leavevmode\vadjust pre{\hypertarget{ref-Wolowicz2006-is}{}}%
Wołowicz, Maciej, Adam Sokołowski, Abdullah Salem Bawazir, and Rafał
Lasota. 2006. {``Effect of Eutrophication on the Distribution and
Ecophysiology of the Mussel Mytilus Trossulus (Bivalvia) in Southern
Baltic Sea (the Gulf of Gda{ń}sk).''} \emph{Limnol. Oceanogr.} 51
(1part2): 580--90.

\leavevmode\vadjust pre{\hypertarget{ref-Wood2011-ui}{}}%
Wood, Simon N. 2011. {``Fast Stable Restricted Maximum Likelihood and
Marginal Likelihood Estimation of Semiparametric Generalized Linear
Models.''} \emph{J. R. Stat. Soc. Series B Stat. Methodol.} 73 (1):
3--36.

\leavevmode\vadjust pre{\hypertarget{ref-Zuur2016-hk}{}}%
Zuur, Alain F, and Elena N Ieno. 2016. {``A Protocol for Conducting and
Presenting Results of Regression-Type Analyses.''} \emph{Methods Ecol.
Evol.} 7 (6): 636--45.

\leavevmode\vadjust pre{\hypertarget{ref-Zuur2010-tn}{}}%
Zuur, Alain F, Elena N Ieno, and Chris S Elphick. 2010. {``A Protocol
for Data Exploration to Avoid Common Statistical Problems.''}
\emph{Methods Ecol. Evol.} 1 (1): 3--14.

\end{CSLReferences}


\bibliographystyle{spphys}
\bibliography{references.bib}

\end{document}
